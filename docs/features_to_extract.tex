\documentclass[a4paper,12pt]{article}
\usepackage{mathtext}
\usepackage[english,russian]{babel}
\usepackage[T2A]{fontenc}	
\usepackage{amssymb} 
\usepackage{amsmath}
\usepackage[utf8]{inputenc}	
\usepackage[unicode, pdftex]{hyperref}
\usepackage{xcolor}
\definecolor{linkcolor}{HTML}{799B03}
\definecolor{urlcolor}{HTML}{799B03} 
\makeatletter
\renewcommand*\env@matrix[1][*\c@MaxMatrixCols c]{%
	\hskip -\arraycolsep
	\let\@ifnextchar\new@ifnextchar
	\array{#1}}
\makeatother
\usepackage[center]{titlesec}

\begin{document}
\subsection*{Общее:}
	
\href{https://github.com/binance-exchange/binance-official-api-docs/blob/master/web-socket-streams.md}{Ссылка на информацию о всех доступных данных.}

Если что, в паре BTC/USDT quote asset - это USDT, base asset - это BTC. Market Taker - тот, кто выставляет рыночный ордер.

\textbf{Крипты:} USDT, BTC - точно, а также варианты: ETH, BCH, LTC, BNB и другие, надо выбрать. Соответственно, всего пар $C_n^2$. В идеале нам нужно, наверное, около четырех крипт.

\textbf{Это не те данные, которые мы непосредственно будем запихивать в нейросеть! Это те данные, которые мы получаем через api, а их обработка - это еще отдельная тема.}

\subsection*{Для каждой пары:}
\begin{itemize}
	\item \textbf{<symbol>@aggTrade и <symbol>@trade:} \\
	Эти потоки в реальном времени кидают информацию о сделках, тут нужно думать, какие фичи можно выделять. Пока могу накидать такие:
	\begin{itemize}
		\item $MinuteQuantity/SecondQuantity$ - объемы продаж в base asset за соответствующую единицу времени.
		\item $MinuteVolume/SecondVolume$ - объем продаж за соответствующую единицу времени в quote asset.
		\item $MinuteNumber/SecondNumber$ - количество сделок за единицу времени.
		\item $SecondMinimumPrice/SecondMaximumPrice/$\\$SecondMeanPrice$ - статистики цены за секунду.
		\item $LastPrice$ - последняя цена.
	\end{itemize}
	А также можно добавить процент сделок, где покупателем выступал Market Taker, процент объема, а еще много чего еще. Дело в том, что мы получаем информацию обо всех сделках за единицу времени, а нам нужно их отобразить в фиксированный набор полей. Предложения принимаются.
	\item \textbf{<symbol>@kline\_<interval>:} \\
	<interval> - это 1m, 3m, 5m, 15m и так далее вплоть, наверное, до 12h. Поток возвращает разные метрики цены за соответствующий интервал. Для каждого интервала:
	\begin{itemize}
		\item $TradeNumber$ - число сделок.
		\item $OpenPrice/ClosePrice/HighPrice/LowPrice$ - статистики цены.
		\item $BaseVolume$ - объем продаж в base asset.
		\item $QuoteVolume$ - объем продаж в quote asset.
		\item $TakerBaseVolume/TakerQuoteVolume$ - объем продаж среди сделок, когда покупателем выступает Market Taker.
	\end{itemize}
	\item \textbf{<symbol>@ticker:}\\
	Этот поток возвращает статистику по паре о последних 24 часах.
	\begin{itemize}
		\item $PriceChange/PriceChangePercent$ - изменение цены.
		\item $WeightedAveragePrice$ - средняя цена.
		\item $FirstPrice/LastPrice$ - цена первой и последней сделок. 
		\item $LastQuantity$ - объем последней сделки.
		\item $BestBidPrice/BestAskPrice$ - цены лучшего бида и аска.
		\item $BestBidQuantity/BestAskQuantity$ - объемы лучшего бида и аска.
		\item $TradeNumber$ - число сделок.
	\end{itemize}
	\item \textbf{<symbol>@depth<levels>@100ms:}\\
	Этот поток каждые 100ms кидает изменения ордербука. <levels> возьмем 10 или 20.
	\begin{itemize}
		\item $BidPrice\text{<}level\text{>}/BidQuantity\text{<}level\text{>}$ - цена и объем ордера продажи, находящегося на определенном уровне ордербука.
		\item $AskPrice\text{<}level\text{>}/AskQuantity\text{<}level\text{>}$ - цена и объем ордеров покупки.
	\end{itemize}
	Тут еще можно докинуть какие-нибудь изменения ордербука за последнюю секунду, например.
\end{itemize}

Как-то так.

\end{document}